\markboth{}{}
% Plus petite marge du bas pour la quatrième de couverture
% Shorter bottom margin for the back cover
\newgeometry{inner=30mm,outer=20mm,top=40mm,bottom=20mm}

%insertion de l'image de fond du dos (resume)
%background image for resume (back)
\backcoverheader

% Switch font style to back cover style
\selectfontbackcover{ % Font style change is limited to this page using braces, just in case

\titleFR{Transport optimal pour l'apprentissage par transfert entre les espaces}

\keywordsFR{Transport Optimal Non-équilibré, Gromov-Wasserstein, Co-Optimal Transport,
Adaptation de Domaine Hétérogène}

\abstractFR{Au cours des dernières années, la remarquable puissance de la théorie du transport optimal
a largement dépassé la comparaison classique des mesures de probabilité évoluant dans le même espace sous-jacent.
Dans cette thèse, nous nous intéressons aux problèmes de transport optimal
entre des espaces incomparables. Plus précisément, nous nous concentrons sur la relaxation marginale
du transport optimal (équilibré) de Gromov-Wasserstein et du Co-Optimal Transport,
ainsi que sur l'intégration de connaissances préalables dans la distance de Gromov-Wasserstein et
sa formulation non-équilibrée. Nous commençons par le Co-Optimal Transport en cadre continu,
qui sert de première étape vers l'étude de l'approximation entropique et
de l'extension non-équilibrée. Ensuite, nous introduisons la formulation non-équilibrée du Co-Optimal Transport
et montrons sa robustesse aux valeurs aberrantes, contrairement à son homologue équilibré. Ensuite,
nous proposons d'utiliser la divergence de Fused Gromov-Wasserstein non-équilibrée
pour aligner les surfaces corticales,
en exploitant simultanément les signaux fonctionnels et la structure anatomique du cerveau humain.
Enfin, nous renforçons davantage la distance de Gromov-Wasserstein avec la capacité de
manipuler plus efficacement les données brutes et d'effectuer un appariement des features génomiques.}

\titleEN{Optimal transport for transfer learning across domains}

\keywordsEN{Unbalanced Optimal Transport, Gromov-Wasserstein, Co-Optimal Transport,
Heterogeneous Domain Adaptation}

\abstractEN{In the recent years, the remarkable versatility of optimal transport theory
has gone far beyond the classic comparison of the probability measures living in the same underlying space.
In this thesis, we are interested in the optimal transport problems
between incomparable spaces. More precisely, we focus on the marginal relaxation of the (balanced)
Gromov-Wasserstein and Co-Optimal Transport,
as well as the integration of prior knownledge into Gromov-Wasserstein distance and
its unbalanced formulation. We start with the Co-Optimal Transport in continuous setting,
which serves as the first step towards the study of the entropic approximation and
unbalanced extension. Then, we introduce the unbalanced formulation of the Co-Optimal Transport
and show its robustness to outliers, by contrast to the balanced counterpart. Next,
we propose to use the fused unbalanced Gromov-Wasserstein divergence to align the cortical surfaces,
by simultaneously exploiting the functional signals and anatomical structure of human brain.
Finally, we further empower the Gromov-Wasserstein distance with the ability to
manipulate more efficiently the input data and to perform meaningful genomic feature matching.}

}

% Rétablit les marges d'origines
% Restore original margin settings
\restoregeometry
