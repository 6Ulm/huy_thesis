\section{Appendix of Chapter 2}

\subsection{Proofs related to Unbalanced Optimal Transport}

%%%%%%%%%%%%%%%%%%%%%%%%%%%%%%%%%%%%%%%%%%%%%%
\begin{proof}[Proof of \Cref{coro:uot_l2_mm}]
Following \citep{Chapel21}, we can write Problem \eqref{uot_l2} as
\begin{align}
    \min_{t \in \bbR^{mn}_{\geq 0}} \langle c, t \rangle + \frac{|| Mt - y ||^2}{2},
\end{align}
where $t = \vect(P), c = \vect(C)$ are the vectorizations of $P, C$, respectively. Here,
\begin{itemize}
    \item[$\bullet$] $M = (\rho_1^{1/2} M_r^T, \rho_2^{1/2} M_c^T, \varepsilon^{1/2} I_{mn})^T \in \bbR^{(m + n + mn) \times (mn)}$.

    \item[$\bullet$] $y = (\rho_1^{1/2} \mu^T, \rho_2^{1/2} \nu^T, \varepsilon^{1/2} \vect(\gamma)^T)^T \in \bbR^{m + n + mn}$.

    \item[$\bullet$] $M_r = \texttt{np.repeat(np.eye(n),m)} \in \bbR^{n \times mn}$.

    \item[$\bullet$] $M_c = [I_m, ..., I_m] = \texttt{np.tile(np.eye(m),n)} \in \bbR^{m \times mn}$.
\end{itemize}
See Appendix A in \citep{Chapel21} for more details of $M_r, M_c$.
Remark that if $A \in \bbR^{m \times n}$ and $B \in \bbR^{m \times p}$, then
\begin{align}
    (A, B) \begin{pmatrix}
        X^T \\
        B^T
    \end{pmatrix}
    = X X^T + B B^T.
\end{align}
Following Equation 23 in \citep{Chapel21},  we have
\begin{align}
    t^{(k+1)}_i = t^{(k)}_i \frac{\max( 0, [M^T y]_i - c_i)}{[M^T M t^{(k)}]_i}.
\end{align}
Now, $\text{mat}(M^T y) = (\rho_1 \mu) \oplus (\rho_2 \nu) + \varepsilon \gamma$.
Here, matrization is the inverse operation of vectorization. Since,
$M^T M = \rho_1 M_r^T M_r + \rho_2 M_c^T M_c + \varepsilon I_{mn}$, we obtain
$\text{mat}(M^T M t^{(k)}) = (\rho_1 P_{\# 1}^{(k)}) \oplus (\rho_2 P_{\# 2}^{(k)})
+ \varepsilon P^{(k)} $. The result then follows.
\end{proof}
%%%%%%%%%%%%%%%%%%%%%%%%%%%%%%%%%%%%%%%%%%%%%%

%%%%%%%%%%%%%%%%%%%%%%%%%%%%%%%%%%%%%%%%%%%%%%
\begin{proof}[Proof of \Cref{prop:uot_minimizer}]
We follow the same proof technique of Lemma 4 in \citep{Khiem20}.
Given a Bregman divergence $D_{\psi}$, for any $t \in \bbR$ such that $tp \in \text{dom}(\psi)$,
we have
\begin{align}
  D_{\psi}(t p | q)
  &= \psi(tp) - \psi(q) - \langle \nabla \psi(q), tp - q \rangle \\
  &= \psi(tp) - \psi(q) - \Big[
    t \langle \nabla \psi(q), p - q \rangle + (t-1) \langle \nabla \psi(q), q \rangle
  \Big] \\
  &= \psi(tp) - \psi(q) + t \Big[ D_{\psi}(p | q) + \psi(q) - \psi(p) \Big]
  - (t-1) \langle \nabla \psi(q), q \rangle \\
  &= t D_{\psi}(p | q) + \big[ \psi(tp) - t \psi(p) \big]
  + (t-1) \big[ \psi(q) - \langle \nabla \psi(q), q \rangle \big].
\end{align}
Denote $g$ the objective function of Problem \eqref{eq:uot_bregman}. We have,
\begin{align}
  g(tP) &= t g(P) +
  \underbrace{\Big( \sum_{k=1}^2 \rho_k \big[ \varphi_k(tP_{\# k}) - t \varphi_k(P_{\# k}) \big]
  + \varepsilon \big[ \varphi(tP) - t \varphi(P) \big] \Big)}_{A(t)} \\
  &+ (t-1) \underbrace{\Big(
    \sum_{k=1}^2 \rho_k \big[ \varphi_k(\mu_k) - \langle \nabla \varphi_k(\mu_k), \mu_k \rangle \big]
    + \varepsilon \big[ \varphi(\gamma) - \langle \nabla \varphi(\gamma), \gamma \rangle \big]
    \Big)}_{B} \\
    &= t g(P) + A(t) + (t-1) B.
\end{align}
For any minimizer $P^* \in \cC$, denote $\cT^* =\{ t \in \bbR: tP^* \in \cC \}$. Clearly, $\cT^*$ is
not empty since $1 \in \cT^*$. Since $P^*$ is a (global) minimizer, we have
$\frac{\partial g(tP^*)}{\partial t} \Big |_{t = 1} = 0$,
or equivalently,
\begin{align}
  g(P^*) &= - B - \frac{\partial A(t)}{\partial t} \bigg|_{t=1} \\
  &= \sum_{k=1}^2 \rho_k \Big[ \langle \nabla \varphi_k(\mu_k), \mu_k \rangle - \varphi_k(\mu_k) \Big]
  + \varepsilon \Big[ \langle \nabla \varphi(\gamma), \gamma \rangle - \varphi(\gamma) \Big] \\
  &+ \sum_{k=1}^2
  \rho_k \Bigg( \varphi_k(P^*_{\# k}) - \frac{\partial \varphi_k(tP^*_{\# k})}{\partial t} \bigg|_{t=1} \Bigg)
  + \varepsilon \Bigg( \varphi(P^*) - \frac{\partial \varphi(tP^*)}{\partial t} \bigg|_{t=1} \Bigg).
\end{align}
The result then follows.
\end{proof}

%%%%%%%%%%%%%%%%%%%%%%%%%%%%%%%%%%%%%%%%%%%%%%
\subsection{Proofs related to Gromov-Wasserstein distance}

%%%%%%%%%%%%%%%%%%%%%%%%%%%%%%%%%%%%%%%%%%%%%%
\begin{corollary}
    The formulations \eqref{GH:zeroth} and \eqref{GH:first} of the Gromov-Hausdorff distance
    are equivalent.
\end{corollary}
\begin{proof}
In the formulation \eqref{GH:first}, by choosing identity mappings (which are clearly isometric embeddings)
$f = \id_X$ and $g = \id_Y$, and $Z = X \cup Y$ equipped with an admissible distance $d$, we have
\begin{equation}
  \gh((X,d_X), (Y,d_Y)) \leq d_{H}^{(Z,d)}(X, Y)
\end{equation}
As this is true for any $d \in \cD(d_X,d_Y)$, we have
$\gh((X,d_X), (Y,d_Y)) \leq \inf_{d} d_{H}^{(X \cup Y, d)}(X, Y)$. For the reverse direction,
given any isometries $f: X \to X'$ and $g: Y \to Y'$ with $X', Y' \subset (Z, d_Z)$
(we call $(X',Y')$ a \textit{metric coupling} \citep{Villani08}), one can define
a distance $d$ on $X \cup Y$ (as shown in Proposition 27.1 in \citep{Villani08}). Then,
\begin{equation}
  \begin{split}
    d_H^{(Z, d_Z)}(f(X), g(Y)) &=
  \max(\sup_{y \in Y} d_Z(g(y), f(X)), \sup_{x \in X} d_Z(f(x), g(Y))) \\
  &= \max(\sup_{y \in Y} d(y, X), \sup_{x \in X} d(x, Y)) \\
  &= d_H^{(X \cup Y, d)}(X, Y) \geq \inf_{d'} d_{H}^{(X \cup Y, d')}(X, Y)
  \end{split}
\end{equation}
We deduce that $\gh((X,d_X), (Y,d_Y)) \geq \inf_{d} d_{H}^{(X \cup Y, d)}(X, Y)$,
thus equality holds.
\end{proof}
%%%%%%%%%%%%%%%%%%%%%%%%%%%%%%%%%%%%%%%%%%%%%%

%%%%%%%%%%%%%%%%%%%%%%%%%%%%%%%%%%%%%%%%%%%%%%
\begin{proof}[Proof of \Cref{lemma:fgw_pgd}]
  Denote $F(P) = \langle C \otimes P, P \rangle + \langle M, P \rangle + \varepsilon \kl(P | \gamma)$.
  Then $\nabla F(P) = C \otimes P + M + \varepsilon \log \frac{P}{\gamma}$.
  The PGD iterate reads: for $\tau > 0$,
  \begin{align}
    P^{(t+1)} = \text{proj}^{\kl}_{U(\mu_x, \mu_y)}
    \left( P^{(t)} \odot e^{-\tau \nabla F(P^{(t)})} \right),
  \end{align}
  where $\text{proj}^{\kl}_{U(\mu_x, \mu_y)}(K) =
  \argmin_{P \in U(\mu_x, \mu_y)} - \varepsilon \langle \log K, P \rangle + \varepsilon H(P)$.
  Denote $\eta = \tau \varepsilon$. We have
  \begin{align}
    \log K &= \log \left( P^{(t)} \odot e^{-\tau \nabla F(P^{(t)})} \right) \\
    &= \log P^{(t)} - \tau (C \otimes P^{(t)} + M)  - \tau \varepsilon \log \frac{P^{(t)}}{\gamma} \\
    &= - \tau (C \otimes P^{(t)} + M) + \log \left( \gamma^{\eta} \odot (P^{(t)})^{1 - \eta} \right).
  \end{align}
  We deduce that
  \begin{align}
    &- \varepsilon \langle \log K, P \rangle + \varepsilon H(P) \\
    &= \eta \langle C \otimes P^{(t)} + M, P \rangle
    - \varepsilon \langle P, \log \left( \gamma^{\eta} \odot (P^{(t)})^{1 - \eta} \right) \rangle
    + \varepsilon H(P) \\
    &= \eta \langle C \otimes P^{(t)} + M, P \rangle +
    \varepsilon \kl \left( P \big | \gamma^{\eta} \odot (P^{(t)})^{1 - \eta} \right)
    - m \left( \gamma^{\eta} \odot (P^{(t)})^{1 - \eta} \right),
  \end{align}
  where $m(Q) = \sum_{i,j} Q_{ij}$ is the mass of measure $Q$. The result then follows.
\end{proof}
%%%%%%%%%%%%%%%%%%%%%%%%%%%%%%%%%%%%%%%%%%%%%%

\subsection{Proofs related to INPUT}
%%%%%%%%%%%%%%%%%%%%%%%%%%%%%%%%%%%%%%%%%%%%%%
\begin{proof}[Proof of \Cref{lemma:convex-smoothness}]
We begin by noticing that $g(\pi) = \kl(\pi | \mu)$ is $1$-strongly convex \wrt the KL divergence.
Indeed, $\nabla g(\pi) = \log \pi - \log \mu$, so for any positive measure $\pi, \gamma$, we have
\begin{align}
    \langle \nabla g(\pi) - \nabla g(\gamma), \pi - \gamma \rangle
    &= \langle
    \log \pi - \log \gamma, \pi - \gamma \rangle.
\end{align}
This means $g$ is $1$-strongly convex \wrt the KL divergence.
To conclude the proof of strong convexity, note that if $f$ is $\sigma$-strongly convex \wrt
some divergence $D$ and $g$ is convex, then the composite function $h = f + g$ is
$\sigma$-strongly convex \wrt $D$. Indeed, for $t \in [0,1]$,
\begin{align}
    h \big( t \pi + (1-t) \gamma \big)
    &= f \big( t \pi + (1-t) \gamma \big) + g \big( t \pi + (1-t) \gamma \big) \\
    &\leq t f(\pi) + (1 - t) f(\gamma) - \sigma \frac{t (1 - t)}{2} D(\pi, \gamma)
    + t g(\pi) + (1 - t) g(\gamma) \\
    &= t h(\pi) + (1 - t) h(\gamma) - \sigma \frac{t (1 - t)}{2} D(\pi, \gamma).
\end{align}
Regarding the relative smoothness, denote $h(\pi) = \kl(\pi_{\#1} | \alpha)$,
then $(\nabla h(\pi))_{ij} = \log (\pi_{\#1})_i - \log \alpha_i$, for every $j$. So,
\begin{equation}
    \begin{split}
        \langle \nabla h(\pi) - \nabla h(\gamma), \pi - \gamma \rangle
    &= \langle
    \log \pi_{\# 1} - \log \gamma_{\# 1}, \pi_{\# 1} - \gamma_{\# 1} \rangle\\
    &= \kl(\pi_{\# 1} | \gamma_{\# 1}) + \kl(\gamma_{\# 1} | \pi_{\# 1}) \\
    &\leq \kl(\pi | \gamma) + \kl(\gamma | \pi) \\
    &= \langle \log \pi - \log \gamma, \pi - \gamma \rangle
    \end{split}
\end{equation}
where the last inequality is due to the data pre-processing inequality.
This means that $h$ is $1$-relatively smooth \wrt the KL divergence.
We conclude that $F$ is $(\lambda_1 + \lambda_2 + \varepsilon)$-relatively smooth.
\end{proof}
%%%%%%%%%%%%%%%%%%%%%%%%%%%%%%%%%%%%%%%%%%%%%%

%%%%%%%%%%%%%%%%%%%%%%%%%%%%%%%%%%%%%%%%%%%%%%
\begin{proof}[Proof of \Cref{prop:convergence-exact-sppa}]
We adapt the proof of Theorem 3.1 in \citep{Lu18}.
First, recall the 4-points identity of Bregman divergence.
\begin{lemma}[Four-point lemma]
    \label{lemma:four-points}
    For any $x, y, z, t \in \text{dom}(\varphi)$, we have
    \begin{equation}
    D_{\varphi}(z, y) - D_{\varphi}(z,x) + D_{\varphi}(t, x) - D_{\varphi}(t, y)
    = \langle \nabla \varphi(x) - \nabla \varphi(y), z - t \rangle.
    \end{equation}
    As a consequence, by taking $y = t$, we obtain the three-point lemma
    \begin{equation}
    D_{\varphi}(z, y) + D_{\varphi}(y, x) - D_{\varphi}(z, x)
    = \langle \nabla \varphi(x) - \nabla \varphi(y), z - y \rangle.
    \end{equation}
\end{lemma}

%%%%%%%%%%%%%%%%%%%%%%%%%%%%%%%%%%%%%%%%%%%%%%%%%%%
\begin{lemma}[Adapted from Lemma 3.2 in \citep{Chen93}]
    \label{lemma:bregman-minimizer-property}
    Suppose $f$ is a $l$-strongly convex, proper differentiable function whose domain is a closed set
    $C$. Let $ x^{(t+1)} = \argmin_{x \in C} f(x) + D_{\varphi}\left(x, x^{(t)} \right)$.
    Then, for any $x \in C$, we have
    \begin{align}
        f(x^{(t+1)}) + (l+1) D_{\varphi}(x, x^{(t+1)}) + D_{\varphi} (x^{(t+1)}, x^{(t)})
        \leq f(x) + D_{\varphi} (x, x^{(t)}).
    \end{align}
    Moreover, if $f$ is also $L$-smooth, then
    \begin{align}
            f(x^{(t+1)}) + (L+1) D_{\varphi}(x, x^{(t+1)}) + D_{\varphi} (x^{(t+1)}, x^{(t)})
            \geq f(x) + D_{\varphi} (x, x^{(t)}).
        \end{align}
\end{lemma}
\begin{proof}[Proof of \Cref{lemma:bregman-minimizer-property}]
    From the optimality of $x^{(t+1)}$, we have
    $\nabla \varphi(x^{(t+1)}) - \nabla \varphi(x^{(t)}) + \nabla f(x^{(t+1)}) = 0$.
    By three-point lemma, for every $x \in C$,
    \begin{align}
        D_{\varphi}(x, x^{(t+1)}) + D_{\varphi} (x^{(t+1)}, x^{(t)}) - D_{\varphi} (x, x^{(t)}) &= \langle \nabla \varphi(x^{(t)}) - \nabla \varphi(x^{(t+1)}), x - x^{(t+1)} \rangle \\
        &= \langle \nabla f(x^{(t+1)}), x - x^{(t+1)} \rangle.
    \end{align}
    As $f$ is $l-$strongly convex and differentiable, we have
    \begin{align}
        f(x) &\geq f(x^{(t+1)}) + \langle \nabla f(x^{(t+1)}), x -
                x^{(t+1)} \rangle  + l D_{\varphi}(x, x^{(t+1)})\\
        &=         f(x^{(t+1)}) + (l+1) D_{\varphi}(x, x^{(t+1)})
        + D_{\varphi} (x^{(t+1)}, x^{(t)}) - D_{\varphi} (x, x^{(t)}).
    \end{align}
    Similarly, if $f$ is $L$-relatively smooth, then
    \begin{align}
        f(x) &\leq f(x^{(t+1)}) + \langle \nabla f(x^{(t+1)}), x -
                x^{(t+1)} \rangle  + L D_{\varphi}(x, x^{(t+1)})\\
        &=         f(x^{(t+1)}) + (L+1) D_{\varphi}(x, x^{(t+1)})
        + D_{\varphi} (x^{(t+1)}, x^{(t)}) - D_{\varphi} (x, x^{(t)}).
    \end{align}
    The result then follows.
\end{proof}
%%%%%%%%%%%%%%%%%%%%%%%%%%%%%

%%%%%%%%%%%%%%%%%%%%%%%%%%%%%
\begin{lemma}
    \label{lemma:bound_seq}
    If $(y_t)_t$ is a nonnegative decreasing sequence, $(x_t)_t$ is a nonnegative sequence and
    $a, b > 0$ such that $y_t \leq a x_{t-1} - b x_t$, then
    $y_t \leq \frac{b-a}{\left( \frac{b}{a} \right)^t - 1} x_0$.
\end{lemma}
\begin{proof}[Proof of \Cref{lemma:bound_seq}]
We have that $\frac{y_t}{b} \leq \frac{a}{b} x_{t-1} - x_t$. So,
$\left( \frac{a}{b} \right)^i \frac{y_{t-i}}{b} \leq
\left(\frac{a}{b} \right)^{i+1} x_{t-i-1} - \left( \frac{a}{b} \right)^i x_{t-i}$.
Computing the telescopic sum, we have
\begin{align}
    \left( \frac{a}{b} \right)^t x_0 - x_t &\geq
    \sum_{i=0}^{t-1} \left( \frac{a}{b} \right)^i \frac{y_{t-i}}{b}
    \geq \frac{y_t}{b} \frac{\left( \frac{a}{b} \right)^t - 1}{\frac{a}{b} - 1}
    = y_t \frac{\left( \frac{a}{b} \right)^t - 1}{a - b}.
\end{align}
Since $x_t \geq 0$, we deduce that
$y_t \leq \frac{b-a}{\left( \frac{b}{a} \right)^t - 1} x_0$.
\end{proof}
%%%%%%%%%%%%%%%%%%%%%%%%%%%%%

%%%%%%%%%%%%%%%%%%%%%%%%%%%%%
\begin{lemma}
\label{lemma:bppa_conv}
    Let $F$ be a differentiable function. Suppose that $F$ is $l-$strongly convex and $L-$ smooth,
    where $L > l \geq 0$. Given a learning rate $l \leq \eta \leq L$,
    define $x^{(t+1)} = \argmin F(x) + \eta D_{\varphi}(x, x^{(t)})$.
    Then, for any initialization $x^{(0)}$, we have
    \begin{align}
        F(x^{(t)}) - F(x^*) \leq \frac{l}{\left( 1 + \frac{Ll }{\eta (L - l)} \right)^t - 1}
        D_{\varphi}(x^*, x^{(0)}).
\end{align}
\end{lemma}
%%%%%%%%%%%%%%%%%%%%%%%%%%%%%
\begin{proof}[Proof of \Cref{lemma:bppa_conv}]
By applying \Cref{lemma:bregman-minimizer-property} with $f = \frac{F}{\eta}$,
which is $\frac{l}{\eta}-$strongly convex, we obtain that
\begin{equation}
    D_{\varphi}(x^{(t+1)}, x^{(t)}) \leq D_{\varphi}(x, x^{(t)})
    - \left( \frac{l}{\eta} + 1 \right) D_{\varphi}(x, x^{(t+1)})
    - \frac{F(x^{(t+1)}) - F(x)}{\eta}.
\end{equation}
In particular, by choosing $x = x^{(t)}$, we have
\begin{equation}
    F(x^{(t+1)}) - F(x^{(t)}) \leq - \eta \left[ \left(\frac{l}{\eta} + 1 \right)
    D_{\varphi}(x^{(t)}, x^{(t+1)}) + D_{\varphi}(x^{(t+1)}, x^{(t)}) \right] \leq 0.
\end{equation}
This means $\big(F(x^{(t)}) \big)_t$ is a (nonnegative) decreasing sequence.
Now, for every $x \in C$ and $l \leq \eta \leq L$,
\begin{align}
    &F(x^{(t+1)}) \\
    &\leq F(x^{(t)}) + \langle \nabla F(x^{(t)}), x^{(t+1)} - x^{(t)} \rangle
    + L D_{\varphi}(x^{(t+1)}, x^{(t)}) \\
    &\leq F(x^{(t)}) + \langle \nabla F(x^{(t)}), x - x^{(t)} \rangle
    + \eta D_{\varphi}(x, x^{(t)}) - \eta D(x, x^{(t+1)})
    + (L - \eta) D_{\varphi}(x^{(t+1)}, x^{(t)}) \\
    &\leq F(x) + (\eta - l) D_{\varphi}(x, x^{(t)})
    - \eta D(x, x^{(t+1)}) + (L - \eta) D_{\varphi}(x^{(t+1)}, x^{(t)}) \\
    &\leq F(x) + (\eta - l) D_{\varphi}(x, x^{(t)}) - \eta D(x, x^{(t+1)}) \\
    &+ (L - \eta) \left[ D_{\varphi}(x, x^{(t)})
    - \left( \frac{l}{\eta} + 1 \right) D_{\varphi}(x, x^{(t+1)})
    - \frac{F(x^{(t+1)}) - F(x)}{\eta} \right] \\
    &= F(x) + (L - l) D_{\varphi}(x, x^{(t)})
    - \left[ L + \frac{l}{\eta}(L - \eta) \right] D_{\varphi}(x, x^{(t+1)})
    - \frac{L - \eta}{\eta} \big[ F(x^{(t+1)}) - F(x) \big],
\end{align}
where the first and third inequalities follow from the strong convexity and
relative smoothness of $F$, respectively. The second one follows by applying
\Cref{lemma:bregman-minimizer-property} with
$f = \frac{1}{\eta} \langle \nabla F(x^{(t)}), \cdot - x^{(t)} \rangle$
(which is $0$-strongly convex). By rearranging the terms, we obtain
\begin{align}
    F(x^{(t+1)}) - F(x) \leq \frac{\eta(L-l)}{L} D_{\varphi}(x, x^{(t)})
    - \left[ l + \frac{\eta (L - l)}{L} \right] D_{\varphi}(x, x^{(t+1)}).
\end{align}
By applying \Cref{lemma:bound_seq} and choosing $x = x^*$, we deduce that
\begin{align}
    F(x^{(t)}) - F(x^*) \leq \frac{l}{\left( 1 + \frac{Ll }{\eta (L - l)} \right)^t - 1}
    D_{\varphi}(x, x^{(0)}).
\end{align}
The result then follows.
\end{proof}
Now, \Cref{prop:convergence-exact-sppa} follows immediately from
\Cref{lemma:bppa_conv,lemma:convex-smoothness}.
\end{proof}
%%%%%%%%%%%%%%%%%%%%%%%%%%%%%%%%%%%%%%%%%%%%%%