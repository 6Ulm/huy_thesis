\section{Appendix of Chapter 6}

%%%%%%%%%%%%%%%%%%%%%%%%%%%%%%%%%%%%%%
\subsection{Additional proof}

\begin{proof}[Proof of \Cref{coro:ugw_ucoot}]
  Let $(P, Q)$ be a solution of the problem $\widetilde{\fugw}(\cX^s, \cX^t)$.
  This means $P_{\#1} = Q_{\#1}, P_{\#2} = Q_{\#2}$. Using the relation
  \begin{align}
    \kl(p \otimes q | a \otimes b) = m(q) \kl(p | a) + m(p) \kl(q | b)
    + \big(m(P) - m(a)\big) \big(m(Q) - m(b)\big),
  \end{align}
  we have,
  \begin{enumerate}
    \item $\kl(P_{\#1} \otimes Q_{\#1} \vert w^s \otimes w^s) =
    \kl(P_{\#1} \otimes P_{\#1} \vert w^s \otimes w^s) =
    \kl(Q_{\#1} \otimes Q_{\#1} \vert w^s \otimes w^s)$.

    \item $\kl(P_{\#2} \otimes Q_{\#2} \vert w^t \otimes w^t) =
  \kl(P_{\#2} \otimes P_{\#2} \vert w^t \otimes w^t) =
  \kl(Q_{\#2} \otimes Q_{\#2} \vert w^t \otimes w^t)$.

    \item $\kl \Big(P \otimes Q | (w^s \otimes w^t) \otimes (w^s \otimes w^t) \Big) =
    \kl \Big(P \otimes P | (w^s \otimes w^t) \otimes (w^s \otimes w^t) \Big)
    = \kl \Big(P \otimes Q | (w^s \otimes w^t) \otimes (w^s \otimes w^t) \Big)$.
  \end{enumerate}
  So, the inequality $\widetilde{\fugw}(\cX^s, \cX^t) \leq \fugw(\cX^s, \cX^t)$ and
  the suboptimality of $P, Q$ with respect to the problem $\fugw(\cX^s, \cX^t)$ imply that
  \begin{equation}
    2 \langle G, P \otimes Q \rangle \leq
      \langle G, P \otimes P \rangle + \langle G, Q \otimes Q \rangle.
  \end{equation}
  Now, with exactly the same proof of \Cref{prop:coot_gw_equiv},
  the above inequality becomes an equality.
\end{proof}

%%%%%%%%%%%%%%%%%%%%%%%%%%%%%%%%%%%%%%
\subsection{Algorithmic details}

\begin{algorithm}[t]
  \caption{Approximation scheme for LB-FUGW}
  \label{alg:lbfugw}
  \begin{algorithmic}[1]
      \STATE \textbf{Input:} $\cX^s, \cX^t, \rho, \alpha, \varepsilon$.
      \STATE \textbf{Output:} Pair of optimal couplings $(P, Q)$.
      \STATE Initialize: $P = Q = w^s \otimes w^t / \sqrt{m(w^s) m(w^t)}$.
      \WHILE{$(P, Q)$ has not converged}
          \STATE Calculate: $c_P = \text{Cost}(P,  G, C, w^s, w^t, \rho, \alpha, \varepsilon)$.
          \STATE Update: $Q \gets \text{Sinkhorn}(c_P, w^s, w^t, \rho m(P), \varepsilon m(P))$.
          % % \Comment{fixed P}
          \STATE Rescale: $Q \gets \sqrt{\frac{m(P)}{m(Q)}} Q$.
          \STATE Calculate: $c_Q = \text{Cost}(Q,  G, C, w^s, w^t, \rho, \alpha, \varepsilon)$.
          \STATE Update: $P \gets \text{Sinkhorn}(c_Q, w^s, w^t, \rho m(Q), \varepsilon m(Q))$.
          % % \Comment{fixed Q}
          \STATE Rescale: $P \gets \sqrt{\frac{m(Q)}{m(P)}} P$.
      \ENDWHILE
  \end{algorithmic}
\end{algorithm}

\begin{algorithm}[h]
  \caption{Sinkhorn algorithm \citep{Sejourne19}}
  \label{alg:sinkhorn}
  \begin{algorithmic}[1]
      \STATE \textbf{Input:} $C, w^s, w^t, \rho, \varepsilon$.
      \STATE \textbf{Output:} Optimal coupling $P$.
      \STATE Initialize dual vectors: $f = 0_n \in \bbR^n, g = 0_p \in \bbR^p$.
      \WHILE{$(f,g)$ has not converged}
          \STATE Update: $f = -\frac{\rho}{\rho + \varepsilon} \log \sum_j \exp \big( g_j + \log w^t_j - \frac{C_{\cdot,j}}{\varepsilon} \big)$.
        \STATE Update: $g = -\frac{\rho}{\rho + \varepsilon} \log \sum_i \exp \big( f_i + \log w^s_i - \frac{C_{i,\cdot}}{\varepsilon} \big)$.
      \ENDWHILE
      \STATE Calculate: $P = (w^s \otimes w^t) \exp \big(f \oplus g - \frac{C}{\varepsilon} \big)$.
  \end{algorithmic}
\end{algorithm}

\begin{algorithm}[h]
  \caption{Cost}
  \label{alg:local_cost}
  \begin{algorithmic}[1]
      \STATE \textbf{Input:} $P, G, C, w^s, w^t, \rho, \alpha, \varepsilon$.
      \STATE \textbf{Output:} Local cost $c$.
      \STATE Calculate: $G \otimes P := \left( \sum_{i,j} G_{i,j,k,l} P_{i,j} \right)_{k,l}$.
      \STATE Calculate:
      \begin{align}
          c = \alpha \; G \otimes P + \frac{1 - \alpha}{2} \; C +
          \rho \; \langle \log \frac{P_{\#1}}{w^s}, P_{\#1} \rangle +
          \rho \; \langle \log \frac{P_{\#2}}{w^t}, P_{\#2} \rangle +
          \varepsilon \; \langle \log \frac{P}{w^s \otimes w^t}, P \rangle.
      \end{align}
  \end{algorithmic}
\end{algorithm}
Here, the notations $\otimes$ and $\oplus$ denote the Kronecker product and sum, respectively.
The exponential, division and logarithm operations are all element-wise.
The scalar product is denoted by $\langle \cdot, \cdot \rangle$.

% First, let us introduce the following problem
% \begin{equation}
%   \text{LB}_1(\cX^s, \cX^t) = \inf_{(P, Q) \in \cE} L_{\theta}(P, Q),
% \end{equation}
% where $\cE = \{(P, Q) \geq 0: P_{\#1} = Q_{\#1}, P_{\#2} = Q_{\#2} \}$ is
% the set of pairs of transportation plans whose corresponding marginal distributions are equal.
% Clearly, we have
% \begin{equation}
%     \text{LB-FUGW} (\cX^s, \cX^t) \leq \text{LB}_1(\cX^s, \cX^t)
%     \leq \fugw(\cX^s, \cX^t).
% \end{equation}
% Denote
% \begin{align}
%   \text{LB}_2 = \inf_{P \in U(Q_{\# 1}, Q_{\# 2})} \inf_{Q \geq 0} L_{\theta}(P, Q)
% \end{align}
% Show that $\text{LB}_2$ has solution?

% Observe that $\cE = \bigcup_{Q \geq 0} \{(P, Q): P \in U(Q_{\# 1}, Q_{\# 2}) \}$.
% Denote $(P^*, Q^*)$ the solution of $\text{LB}_1$. Then
% \begin{align}
%   \text{LB}_1(\cX^s, \cX^t) \leq
% \end{align}
% We have, for every $P, Q \geq 0$ and $t > 0$,
% \begin{align}
%   g(t(P, Q)) &= t g(P, Q) + (1-t) \left[ \lambda_1 m(\mu_X)^2 + \lambda_2 m(\mu_Y)^2 \right]
%   + (\lambda_1 + \lambda_2) t \log t m(P) m(Q) \\
%   &= t g(P, Q) + (1 - t) K + \lambda t \log t m(P) m(Q).
% \end{align}
% So,
% \begin{align}
%   \min_{t > 0} g(t(P, Q)) = K - t^* \lambda m(P) m(Q)
% \end{align}
% where $t^* = \exp \left( \frac{K - g(P, Q)}{\lambda m(P) m(Q)} - 1 \right)$.

%%%%%%%%%%%%%%%%%%%%%%%%%%%%%%%%%%%%%%
\subsection{Dataset description}
%
The presented experiments rely on the Individual Brain Charting (IBC) dataset.
%
A detailed description of the preprocessing pipeline of the IBC data is provided
in \citep{Pinho2021}.
%
Raw data were preprocessed using \emph{PyPreprocess}
\footnote{\url{https://github.com/neurospin/pypreprocess}}.

All fMRI images, i.e. GE-EPI volumes, were collected twice with reversed
phase-encoding directions, resulting in pairs of images with distortions going
in opposite directions.
%
Susceptibility-induced off-resonance field was estimated from the two Spin-Echo
EPI volumes in reversed phase-encoding directions.
%
The images were corrected based on the estimated deformation model.
%
Details about the method can be found in \citep{Andersson2003}.

Further, the GE-EPI volumes were aligned to each other within every participant.
%
A rigid-body transformation was employed, in which the average volume of all
images was used as reference \citep{Friston1995}.
%
The anatomical and motion-corrected fMRI images were given as input to
\emph{FreeSurfer} v6.0.0, in order to extract meshes of the tissue
interfaces and the sampling of functional activation on these meshes, as
described in \citep{vanessen2012}.
%
The corresponding maps were then resampled to the fsaverage7 (high resolution,
163k nodes per hemisphere) and fsaverage5 (low resolution, 10k nodes per hemisphere) templates of
FreeSurfer \citep{Fischl1999}.

FMRI data were analyzed using the \textit{General Linear Model}.
%
Regressors of the model were designed to capture variations in BOLD response
strictly following stimulus timing specifications.
%
They were estimated through the convolution of boxcar functions, that represent
per-condition stimulus occurrences, with the canonical \textit{Hemodynamic
  Response Function} (HRF).
%
To build such models, paradigm descriptors grouped in triplets (i.e. onset time,
duration and trial type) according to BIDS Specification were determined from
the log files' registries generated by the stimulus-delivery software.
%
To account for small fluctuations in the latency of the HRF peak response,
additional regressors were computed based on the convolution of the same
task-conditions profile with the time derivative of the HRF.
%
Nuisance regressors were also added to the design matrix in order to minimize
the final residual error.
%
To remove signal variance associated with spurious effects arising from
movements, six temporal regressors were defined for the motion parameters.
%
Further, the first five principal components of the signal, extracted from
voxels showing the $5\%$ highest variance, were also regressed to capture
physiological noise \citep{Behzadi2007}.

In addition, a discrete-cosine basis was included for high-pass filtering
(\textit{cutoff}=$\frac{1}{128}\textrm{Hz}$).
%
Model specification was implemented using \textit{Nilearn} v0.8.1
\citep{Abraham2014}, a Python library for statistical learning on neuroimaging
data (\url{https://nilearn.github.io}).