
%%%%%%%%%%%%%%%%%%%%%%%%%%%%%%%%%%%%%%%%%%%%%%
\section{Inexact Bregman Proximal Point for Entropic Unbalanced Optimal Transport}
\label{sec:input}

\subsection{Motivation and algorithm}

While usually approximated via the entropic regularization, the OT distance can also be directly
estimated using the Bregman proximal point (BPP) method \citep{Chen93,Teboulle97}.
Though this scheme can apply to the class of Bregman divergences,
we will exclusively focus on the KL divergence throughout this thesis.
More precisely, for fixed learning rate $\eta > 0$, at iteration $t$, we solve
\begin{align}
  \label{eq:bpp_eot}
  P^{(t+1)} = \argmin_{P \in U(\mu, \nu)} \langle C, P \rangle + \eta \; \kl(P | P^{(t)}).
\end{align}
This is a slightly modified entropic OT problem and can be solved with the Sinkhorn algorithm,
ut now the OT plan is computed by
\begin{align}
  P^{(t+1)} = P^{(t)} \odot \exp \left( \frac{f^{(t+1)} \oplus g^{(t+1)} - C}{\eta} \right),
\end{align}
where $(f^{(t+1)}, g^{(t+1)})$ is the optimal dual vectors of Problem \eqref{eq:bpp_eot}.
Classic results \citep{Chen93} guarantee the linear convergence of BPP method
to the minimum and the minimizer of the OT problem. Interestingly, this scheme can be easily extended to the
generic UOT problem \eqref{eq:discrete_ent_uot}, where each iteration boils down to solving
\begin{align}
\label{eq:bpp_uot}
  &P^{(t+1)} \\
  &= \; \argmin_{P \in \bbR^{m \times n}_{\geq 0}}
  \langle C, P \rangle + \rho_1 \kl(P_{\# 1} | \mu)
  + \rho_2 \kl(P_{\# 2} | \nu) + \varepsilon \kl(P | \gamma) + \eta \kl(P | P^{(t)}) \\
  &= \; \argmin_{P \in \bbR^{m \times n}_{\geq 0}}
  \left \langle C - \eta \log \frac{P^{(t)}}{\gamma}, P \right \rangle
  + \rho_1 \kl(P_{\# 1} | \mu) + \rho_2 \kl(P_{\# 2} | \nu) + (\varepsilon + \eta) \kl(P | \gamma).
\end{align}
This is nothing but an entropic UOT problem with modified cost and regularization.
Thus, any solvers discussed in \Cref{sub:uot_optim} can be used. In particular,
by construction, BPP is naturally applicable to the unregularized UOT.

Interestingly, it is not necessarily to solve exactly the entropic OT problem \eqref{eq:bpp_eot},
but even as few as one Sinkhorn iteration can work in practice.
This approach, known as \textit{Inexact Proximal OT}, is introduced in \citep{Xie20}.
While the inexact BPP scheme has recently been applied to the balanced OT \citep{Xie20,Yang22},
we are not aware of any extension to the unbalanced counterpart.
Our proposed method, called \textit{INexact Proximal Unbalanced optimal Transport} (INPUT),
follows directly from \citep{Xie20}, since it is simple to implement and
usually performs well in practice. The algorithmic details of INPUT can be found in \Cref{alg:isppa}.
\begin{algorithm}[t]
  \caption{INPUT algorithm for Problem \eqref{eq:discrete_ent_uot}.}
  \label{alg:isppa}
\begin{algorithmic}[1]
  \STATE \textbf{Input:} cost matrix $C \in \bbR^{m \times n}$,
  measures $\mu \in \bbR^m_{> 0}, \nu \in \bbR^n_{> 0}, \gamma \in \bbR^{m \times n}_{> 0}$,
  regularization $\varepsilon \geq 0$, relaxation parameters $\rho_1, \rho_2 > 0$,
  learning rate $\eta > 0$.
  \FOR{$t=1, \dots, T$}
  \STATE Calculate new cost: $C^{(t+1)} \gets C - \eta \log \Big( \frac{P^{(t)}}{\gamma} \Big)$.
  \STATE Solve approximatively the entropic UOT problem
  \begin{align}
    P^{(t+1)} \approx \argmin_{P \geq 0} \; \langle C^{(t+1)}, P \rangle +
    \rho_1 \kl(P_{\# 1} | \mu) + \rho_2 \kl(P_{\# 2} | \nu) + (\varepsilon + \eta) \kl(P | \gamma).
  \end{align}
  \ENDFOR
  \STATE \textbf{Output:} transport plan $P^{(T)}$.
\end{algorithmic}
\end{algorithm}

\subsection{Convergence analysis in the exact setting}

In practice, performing the exact BPP iteration \eqref{eq:bpp_uot}
may not be an efficient way to solve Problem \eqref{eq:discrete_ent_uot} since
solving exactly many entropic subproblems can be computationally expensive,
However, for theoritical interest, let us first start with the convergence analysis of this scheme.
Thanks to Lemma 3.3 and Theorem 3.4 in \citep{Chen93}, if $P^*$ is a minimizer of
Problem \eqref{eq:discrete_ent_uot}, then
\begin{enumerate}
  \item The sequence $\big( \kl(P^*, P^{(t)}) \big)_t$ is nonincreasing and converges to $0$.
  \item The sequence $\big( F(P^{(t)}) \big)_t$ is nonincreasing and
  \begin{align}
    \label{eq:chen_teboulle}
    F(P^{(t)}) - F(P^*) \leq \frac{\eta}{t} \kl(P^* | P^{(0)}).
  \end{align}
\end{enumerate}
The upper bound \eqref{eq:chen_teboulle} can be further improved by exploiting the structure of
the objective function of Problem \eqref{eq:discrete_ent_uot}.
First, we introduce the notions of relative smoothness \citep{Bauschke17} and
strong convexity \citep{Lu18} of a function with respect to the KL divergence.
% \begin{definition}[Bregman proximal point algorithm]
%   Given a nonepmpty closed convex set of $E \subset \bbR^d_{\geq 0}$
%   and a proper closed convex function $f: E \to \bbR$, consider the following
%   convex optimization problem
%   \begin{align}
%     \min_{x \in E} f(x).
%   \end{align}
%   For learning rate $\eta > 0$, the exact BPP scheme with respect to the KL divergence reads
%   \begin{align}
%     \label{eq:bppa}
%     x^{(t+1)} = \argmin_{x \in E} f(x) + \eta \kl(x | x^{(t)}),
%   \end{align}
%   and the inexact BPP scheme reads
%   \begin{align}
%     \label{eq:bppa_inexact}
%     x^{(t+1)} \approx \argmin_{x \in E} f(x) + \eta \kl(x | x^{(t)}).
%   \end{align}
%   Here, $x^{(t+1)}$ is only an approximate solution in some predefined sense.
% \end{definition}
\begin{definition}
  \label{def:smooth-convex}
  Let $f:C \to \bbR$ be a differentiable convex function, where $C \subset \bbR^d_{\geq 0}$ is
  a nonepmpty closed convex set. Given $L \geq 0$, we say that $f$ is $L-$smooth relative to
  the KL divergence if for any $x, y \in C$,
  \begin{align}
    f(x) \leq f(y) + \langle \nabla f(y), x - y \rangle + L \; \kl(x | y).
  \end{align}
  Given $l \geq 0$, we say $f$ is $l-$strongly convex relative to the KL divergence
  if for any $x, y \in C$,
  \begin{align}
    f(x) \geq f(y) + \langle \nabla f(y), x - y \rangle + l\; \kl(x, y).
  \end{align}
\end{definition}
Recall that the objective function of Problem \eqref{eq:discrete_ent_uot} is
$F(P) = \langle C, P \rangle + \rho_1 \kl(P_{\# 1} | \mu)
+ \rho_2 \kl(P_{\# 2} | \nu) + \varepsilon \kl(P | \gamma)$. Then, we have
\begin{lemma}
  \label{lemma:convex-smoothness}
  $F$ is $\varepsilon$-strongly convex and $(\rho_1 + \rho_2 + \varepsilon)$-relatively smooth
  with respect to the KL divergence.
\end{lemma}
The following result is a simple generalization of Theorem 3.1 in \citep{Lu18}.
\begin{proposition}[Convergence rate of exact BPP for Problem \eqref{eq:discrete_ent_uot}]
  \label{prop:convergence-exact-sppa}
  For every $\eta > 0$, exact BPP scheme decreases the value of $F(\cdot)$ with each iteration $t$:
  the sequence $\big( F(\pi^{(t)}) \big)_t$ is monotonically decreasing.
  Moreover, if $\varepsilon \leq \eta \leq \rho_1 + \rho_2 + \varepsilon$,
  then we have
  \begin{align}
    F(\pi^{(t)}) - F(\pi^*)
    \leq \frac{\varepsilon}{\left( 1 +
    \frac{\varepsilon (\rho_1 + \rho_2 + \varepsilon)}{\eta (\rho_1 + \rho_2)} \right)^t - 1}
    \kl(\pi^* | \pi^{(0)}).
  \end{align}
\end{proposition}
It is not difficult to check that this bound is weaker than the one in
Inequality (\ref{eq:chen_teboulle}). Indeed,
\begin{align}
  \frac{\varepsilon}{\left( 1 +
  \frac{\varepsilon (\rho_1 + \rho_2 + \varepsilon)}{\eta (\rho_1 + \rho_2)} \right)^t - 1}
  \leq \frac{\varepsilon}{\frac{t \varepsilon(\rho_1 + \rho_2 + \varepsilon)}{\eta (\rho_1 + \rho_2)}}
  = \frac{\eta}{t} \frac{\rho_1 + \rho_2}{\rho_1 + \rho_2 + \varepsilon}
  \leq \frac{\eta}{t}.
\end{align}

\subsection{Convergence analysis in the inexact setting}

Despite the simplicity, it is difficult to study the convergence of INPUT.
In particular, while it is an immediate extension of the work of \citet{Xie20} on the balanced OT,
their proof techniques of the convergence can not be adapted to the unbalanced setting.
This is because they rely on the property of the set of admissible couplings,
which is not available in the UOT. Moreover, their assumptions and conditions are also
not trivial to verify in practice, thus the convergence results are mostly of theoretical interest.

The convergence analysis of the inexact BPP has already been studied at the same time
as the exact one. Typically, one can control the approximation error using
$\varepsilon$-subdifferential \citep{Burachik97,Kiwiel97},
or bounded subgradient \citep{Eckstein98,Rockafellar76}, to name a few.
We are studying the literature in this domain to identify the relevant criteria, which are
amenable to study the convergence and to verify in practice.

\subsection{Illustration on toy example}

\paragraph{Experimental setup}
We consider a synthetic dataset:
the source data $X$ contains $200$ points forming an ellipse and a square,
assigned with the same uniform probability on both shapes
$\mu = \frac{1}{200} \sum_{i=1}^{200} \delta_{x_i}$. The target data $Y$ also contains $200$ points
forming an ellipse and a circle, assigned with the histogram
$\nu = \frac{3}{200} \sum_{j=1}^{30} \delta_{y_j \in \text{Circle}} +
\frac{7}{200} \sum_{j=31}^{100} \delta_{y_j \in \text{Circle}} +
\frac{7}{200} \sum_{j=1}^{100} \delta_{y_j \in \text{Ellipse}}$.
The objective in this experiment is to estimate the entropic UOT cost, where
we choose $\gamma = \mu \otimes \nu$ and the cost $C(x, y) = || x - y||^2_2$.

\paragraph{Competing methods}
For INPUT, we consider $3$ versions corresponding to $3$ solvers for the inner entropic UOT problem:
\texttt{INPUT-Sinkhorn}, \texttt{INPUT-TI-v2}, \texttt{INPUT-MM}.
Here, we choose the variant of Sinkhorn-TI (\Cref{alg:TI_Sinkhorn_variant})
since it is more simple to implement, yet appears to perform comparably to the Sinkhorn-TI.
The Sinkhorn-based methods include
\texttt{Sinkhorn} (\Cref{alg:Sinkhorn_algo}), \texttt{Sinkhorn-TI-v1} (\Cref{alg:TI_Sinkhorn})
and its variant \texttt{Sinkhorn-TI-v2} (\Cref{alg:TI_Sinkhorn_variant}).
We emphasize that all Sinkhorn-based methods require log-domain implementation
for numerical stability, whereas INPUT does not suffer this issue,
thus is implemented with direct vector-matrix multiplication.
Apart from these $2$ families of solvers, we also evaluate the performance of
Majorization-Minimization algorithm \texttt{MM}.

\paragraph{Results}
We set up $4$ scenarios to verify if INPUT can overcome the limitations of other existing methods.
More precisely, the first one considers the situation of small relaxations and large regularization.
It is an easy test and we expect that all methods perform well.
In the second one, we fix the relaxations but choose small regularization
in order to hinder convergence of Sinkhorn-based methods.
The third scenario uses fixed regularization but very large marginal relaxations to slow down
the convergence of MM.
The last one combines small regularization and very large relaxations. It is designed to challenge
both Sinkhorn and MM methods.

It is clear that the INPUT family consistently and significantly outperforms other solvers,
The distinction becomes even more visible in the regimes where Sinkhorn and MM struggle.
Except for the first scenario, by contrast to its competitors,
it seems that INPUT does not require compensating the running time for the quality of the estimation.
We also observe that, within the family of INPUT,
combining INPUT with Sinkhorn-TI yields the most efficient algorithm.
Amongst the Sinkhorn-based solvers, Sinkhorn-TI shows clear improvement over Sinkhorn,
even though this advantage quickly diminishes when regularization is small.

\begin{table}[]
  \small
  \centering
  \begin{tabular}{|cc|c|c|c|c|}
  \hline
  \multicolumn{2}{|c|}{} &
    \textbf{\begin{tabular}[c]{@{}c@{}}Scenario 1\\ $\rho_1 = 40$\\ $\rho_2 = 50$\\ $\varepsilon = 1$\end{tabular}} &
    \textbf{\begin{tabular}[c]{@{}c@{}}Scenario 2\\ $\rho_1 = 40$\\ $\rho_2 = 50$\\ $\varepsilon = 1e-3$\end{tabular}} &
    \textbf{\begin{tabular}[c]{@{}c@{}}Scenario 3\\ $\rho_1 = 4000$\\ $\rho_2 = 5000$\\ $\varepsilon = 1$\end{tabular}} &
    \textbf{\begin{tabular}[c]{@{}c@{}}Scenario 4\\ $\rho_1 = 4000$\\ $\rho_2 = 5000$\\ $\varepsilon = 1e-3$\end{tabular}} \\ \hline
  \multicolumn{1}{|c|}{\multirow{3}{*}{\textbf{\begin{tabular}[c]{@{}c@{}}Sinkhorn\\ family\end{tabular}}}} &
    \textbf{Sinkhorn} &
    \begin{tabular}[c]{@{}c@{}}0.275 $\pm$ 0.011\\ {\color{blue}{{41.618}}} \end{tabular} &
    \begin{tabular}[c]{@{}c@{}}55.946 $\pm$ 5.893\\ 40.588\end{tabular} &
    \begin{tabular}[c]{@{}c@{}}{\color{red}{{0.020 $\pm$ 0.002}}} \\ 57.667\end{tabular} &
    \begin{tabular}[c]{@{}c@{}}No convergence at \\ this tolerance\end{tabular} \\ \cline{2-6}
  \multicolumn{1}{|c|}{} &
    \textbf{TI-v1} &
    \begin{tabular}[c]{@{}c@{}}0.389 $\pm$ 0.018\\ {\color{blue}{{41.618}}}\end{tabular} &
    \begin{tabular}[c]{@{}c@{}}37.679 $\pm$ 0.926\\ {\color{red}{{40.576}}}\end{tabular} &
    \begin{tabular}[c]{@{}c@{}}1.118 $\pm$ 0.047\\ 57.613\end{tabular} &
    \begin{tabular}[c]{@{}c@{}}31.088 $\pm$ 0.505\\ 56.834\end{tabular} \\ \cline{2-6}
  \multicolumn{1}{|c|}{} &
    \textbf{TI-v2} &
    \begin{tabular}[c]{@{}c@{}}0.318 $\pm$ 0.013\\ {\color{blue}{{41.618}}}\end{tabular} &
    \begin{tabular}[c]{@{}c@{}}31.679 $\pm$ 2.919\\ {\color{red}{{40.576}}}\end{tabular} &
    \begin{tabular}[c]{@{}c@{}}0.871 $\pm$ 0.034\\ 57.613\end{tabular} &
    \begin{tabular}[c]{@{}c@{}}24.828 $\pm$ 0.243\\ 56.834\end{tabular} \\ \hline
  \multicolumn{1}{|c|}{\multirow{3}{*}{\textbf{\begin{tabular}[c]{@{}c@{}}INPUT\\ family\end{tabular}}}} &
    \textbf{Sinkhorn} &
    \begin{tabular}[c]{@{}c@{}}0.190 $\pm$ 0.008\\ {\color{blue}{{41.618}}}\end{tabular} &
    \begin{tabular}[c]{@{}c@{}}0.676 $\pm$ 0.032\\ {\color{blue}{{40.521}}}\end{tabular} &
    \begin{tabular}[c]{@{}c@{}}1.143 $\pm$ 0.048\\ {\color{blue}{{57.608}}}\end{tabular} &
    \begin{tabular}[c]{@{}c@{}}1.387 $\pm$ 0.026\\ {\color{blue}{{55.866}}}\end{tabular} \\ \cline{2-6}
  \multicolumn{1}{|c|}{} &
    \textbf{TI-v2} &
    \begin{tabular}[c]{@{}c@{}}{\color{red}{{0.149 $\pm$ 0.008}}}\\ {\color{blue}{{41.618}}} \end{tabular} &
    \begin{tabular}[c]{@{}c@{}}{\color{red}{{0.649 $\pm$ 0.013}}}\\ {\color{blue}{{40.521}}} \end{tabular} &
    \begin{tabular}[c]{@{}c@{}}{\color{blue}{{0.018 $\pm$ 0.002}}} \\ {\color{red}{{57.609}}}\end{tabular} &
    \begin{tabular}[c]{@{}c@{}}{\color{blue}{{0.079 $\pm$ 0.010}}} \\ {\color{red}{{55.868}}}\end{tabular} \\ \cline{2-6}
  \multicolumn{1}{|c|}{} &
    \textbf{MM} &
    \begin{tabular}[c]{@{}c@{}} {\color{blue}{{0.068 $\pm$ 0.006}}} \\ 41.627\end{tabular} &
    \begin{tabular}[c]{@{}c@{}} {\color{blue}{{0.186 $\pm$ 0.007}}} \\ 40.640\end{tabular} &
    \begin{tabular}[c]{@{}c@{}}1.038 $\pm$ 0.026\\ 57.738\end{tabular} &
    \begin{tabular}[c]{@{}c@{}}1.028 $\pm$ 0.021\\ 56.723\end{tabular} \\ \hline
  \multicolumn{2}{|c|}{\textbf{MM}} &
    \begin{tabular}[c]{@{}c@{}}0.193 $\pm$ 0.010\\ {\color{red}{{41.621}}} \end{tabular} &
    \begin{tabular}[c]{@{}c@{}}0.864 $\pm$ 0.024\\ 40.558\end{tabular} &
    \begin{tabular}[c]{@{}c@{}}0.726 $\pm$ 0.046\\ 59.089\end{tabular} &
    \begin{tabular}[c]{@{}c@{}}{\color{red}{{0.944 $\pm$ 0.026}}} \\ 57.829\end{tabular} \\ \hline
  \end{tabular}
  \caption{Time $\pm$ standard deviation (in seconds) required to reach the predefined tolerance
  (first line) and entropic UOT cost (second line). The lower score the better.
  {\color{blue}{{Blue number}}} indicates the best score.
  {\color{red}{{Red number}}} indicates the second best score.
  \label{t:uot_time_compare}}
\end{table}

\subsection{Discussion}

While this experiment is mainly for the proof-of-concept purpose,
it shows that INPUT is a promising alternative solver for the UOT problem.

\paragraph{Strengths of INPUT}
There are two very appealing features. First, INPUT overcomes the limitations of
the MM and Sinkhorn-based algorithms. In particular,
it can, not only handle both balanced/unbalanced, and unregularized/regularized settings,
but also, empirically, can converge very fast to the optimal plan and the global minimum,
even in the regimes of very small regularization (where Sinkhorn-based methods struggles),
or of very large relaxation (where MM converges slowly).
We summarize the applicability of these algorithms in \Cref{t:uot_algo_compare}.

Second, the presence of learning rate $\eta$
increases the level of regularization in the inner entropic UOT subproblem,
thus brings two important benefits. The first one is on the reduction of number of iterations:
the larger the regularization, the faster the Sinkhorn algorithm converges. As a consequence,
running only a few iterations is usually enough to obtain a decent approximation of
the true solution. The second advantage is on the acceleration per BPP iteration. In practice,
when the regularization is not too small, one can ignore the log-domain implementation
and employ the one with direct vector-matrix multiplication,
without any concern about the numerical overflow issue.
As a result, this allows to speed up the calculation of the iterates.

\paragraph{Weaknesses of INPUT} There is no free lunch. INPUT has two drawbacks.
First, the cost matrix must be recalculated at the beginning of each BPP iteration.
This can be computationally expensive and prevents INPUT from being scalable.
Second, while tuning the learning rate is neither too tricky nor difficult,
it may take some effort to find an appropriate value. One possible workaround,
which usually works well in practice, is to start with $\eta$ small and not too far from $\varepsilon$.
The intuition of this heuristic comes from \Cref{prop:convergence-exact-sppa} of exact BPP scheme,
which indicates that, for fixed initialization, the smaller the learning rate, the smaller
the potential gap between the estimation and the minimum.
\begin{table}[t]
	\centering
		\begin{tabular}{|l|c|c|c|c|}
    \hline
    & \textbf{\makecell{Scaling}}
    & \textbf{\makecell{TI-Sinkhorn}} & \textbf{MM} & \textbf{INPUT} \\
    \hline
    \makecell[l]{Unregularized \\ setting} & \nomark & \nomark & \yesmark & \yesmark \\
    \hline
    \makecell[l]{Balanced \\ setting} & \yesmark & \yesmark & \nomark  & \yesmark  \\
    \hline
    \makecell[l]{Semi-relaxed \\ setting} & \yesmark & \yesmark & \nomark  & \yesmark  \\
    \hline
    \makecell[l]{Major \\ drawbacks} & \makecell{Very slow conv. \\ for small $\varepsilon$}
    & \makecell{Slow conv. \\ for small $\varepsilon$}
    & \makecell{Slow conv. \\ for large $\rho$}
    & \makecell{Cost recalculation, \\
    extra tuning of \\ hyperparameters} \\
    \hline
		\end{tabular}
		\caption{Summary of features of algorithms for entropic UOT problem \eqref{eq:discrete_ent_uot}.
    Unregularized setting refers to $\varepsilon = 0$, balanced setting corresponds to
    $\rho_1 = \rho_2 = \infty$ and semi-relaxed setting refers to either $\rho_1 = \infty$
    or $\rho_2 = \infty$.
    \label{t:uot_algo_compare}}
\end{table}

\paragraph{Can we adapt INPUT to solve the squared $l_2$-regularized Problem \eqref{uot_l2} ?}
In theory, yes, since the squared $l_2$-norm is a Bregman divergence,
thus the (inexact) BPP scheme is applicable. However, comparing to the MM solver,
we doubt that it would bring any gain in convergence speed. Indeed,
recall that the main motivation of INPUT comes from the poor convergence behavior of Sinkhorn
algorithm when \textbf{regularization} is small. For this reason, adding more regularization
to each BPP iteration helps accelerating the calculation
and improving the convergence of the algorithm. By contrast,
the case of squared $l_2$-norm does not suffer the same issue,
but rather on the \textbf{relaxation} parameters. So, more regularization does not help.